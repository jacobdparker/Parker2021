\chapter{Final Conclusions}\label{conclusion}

The crowning achievement of my graduate work is the success of the ESIS sounding rocket mission, which I most certainly did not accomplish alone.
ESIS succeeded in measuring plasma velocities, in multiple temperature regimes of the solar atmosphere, over a large field of view, every 10 seconds during its 5 minutes of observing time.
Even on an exceptionally quiet day on the sun, we captured tens of small solar eruptions and found their velocities to have interesting spatial distribution and temporal evolution at 10 seconds timescales.
This demonstrates that the simplifying assumptions we often make when analyzing solar data to account for missing information need to be continuously challenged.
It also supports the utility and importance of CTIS instruments for solar study, and the need to build better methods to process the sometimes ambiguous and difficult to analyze data.

The April 18 flare analyzed in the first chapter has been interpreted differently by other scientists and is a great example of when a CTIS observation would have been useful.
If it is indeed a tearing mode propagating along a current sheet that created the observed wave in the flare ribbon, then the oscillations in space and velocity should repeat along the ribbon, as is assumed in our model.
Fortunately the high resolution slit jaw imager on IRIS allowed us to see the oscillation in space.
Unfortunately, because IRIS is a slit spectrograph and was operating in sit and stare mode, we only have velocity information at one position.
Assuming extreme luck and perfect pointing of IRIS during the flare it would have been possible to raster over the ribbon and measure the velocity at multiple positions, however the flare ribbon evolved much too quickly in time and we still would have been left with an incomplete picture of the velocity distribution.
While the velocity information would have been difficult to retrieve from an ESIS style observation of this event, at least the increased field of view and spectral information in every pixel make it possible to gain a more complete picture of the ribbons evolution.

The first flight of MOSES was also successful at capturing velocity information over its entire field of view in a single wavelength.
Data from that flight demonstrated that CTIS data could be inverted to return velocity information from spatially extended eruptions \citep{Fox2010}.
It also inspired several design changes implemented in ESIS \citep{ESIS} that led to higher quality and more easily interpretable data.
Differing point spread functions in each channel of MOSES negatively effected the velocity resolution and made interpreting small explosive events difficult \citep{Fox2010,Rust2019}.
This inspired dedicated and independently focusable gratings in channels with more similar point spread functions, making it much easier to compare channels, even visually, at small scales.
The detection of significant unexpected spectral content in the MOSES images led to the addition of a field stop in ESIS, clearly defining the field of view to be the same in every wavelength.
With out the field stop, it would have been incredibly difficult to differentiate intensity from \ov, \mgxbright, and \hei\ without a complete inversion of the data.
Since information from each wavelength only partially overlaps we can already tell we captured certain events in multiple wavelengths, demonstrating the incredible amount of information gathered by ESIS every ten seconds. 

A very exciting study yet to be done with the ESIS data is to invert the spatially extended jet (possible minifilament eruption) in each of the bright wavelengths so that we can track the plasma through the transition regions and get a three dimensional picture of the velocity distribution.
We know already that the velocity distribution of the jet is evolving on ten seconds timescales from examining the \ov\ difference movies.
Therefore ESIS managed to capture this jet in a way that no currently operating solar instrument, or even combination of instruments, could have, giving us a truly one of a kind picture of the jet.
Successful implementation of this study will show the full potential of ESIS and will hopefully help inform and clarify current theories and models of jet formation and evolution.

Though the first launch of ESIS was incredibly successful, we can only learn so much about the sun from 5 minutes of data.
The sounding rocket flights of MOSES and ESIS have allowed us to improve our instrumentation and analysis techniques, but have yet to significantly improve out understanding of the sun.
There has been increased interest and popularity of this style of instrument recently, with the proposals of the MUSE\citep{MUSE} and COSIE\citep{winebarger2019} satellites, the flight of the MaGIXS\citep{MaGIXS} sounding rocket with a slot, and the use of EIS slot data for scientific studies\citep{harra2017,harra2020}.
This gives me hope that the community is almost convinced that we are capable of understanding, and are seeing the potential in, overlapogram style data, as well as more openly acknowledging the shortcomings of current observing methods.
I hope to continue to contribute to the field of solar physics by working on CTIS style instruments and their data, so that we can get this technology on a larger platform.
That way we can truly unlock the next level of solar studies and improve our understanding of this giant ball of gas that affects us on earth every second, of every day.



 