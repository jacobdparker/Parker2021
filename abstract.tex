% Abstract 
The solar atmosphere is an energetic and violent place capable of producing eruptions that affect us on earth.
In order to better understand these events, so that we might improve out ability to model and predict them, we observe the sun from space to diagnose the local plasma conditions and track its evolution.
The transition region, a thin region of the solar atmosphere separating the chromosphere from the corona, is where the solar atmosphere transitions rapidly from ten thousand, to one million kelvin and is therefore thought to play an important roll in the transfer of mass and energy to the hot corona.
The sun's magnetic field, and magnetic reconnection, are thought to contribute to the increased temperature of the corona, since the cooler lower solar atmosphere cannot heat it \jdp{\sout{via thermal processes}}.
Explosive events, small solar eruptions likely driven by magnetic reconnection, are frequent in the transition region, making it an attractive area of the atmosphere to study and gather information on the processes.
Using Computed Tomography Imaging Spectrographs (CTIS), capable of measuring spectral line profiles over a wide fields of view at every exposure, we find many eruptive events in the transition region to be spatially complex, three dimensional, and to evolve on rapid timescales.
This demonstrates the utility of, and need to continue developing, CTIS style instruments for solar study since they provide a more complete picture of solar events, allowing us to improve our understanding of our closest star.

